\documentclass[a4paper,12pt]{article}
\usepackage{fancyhdr}
\usepackage{fancyheadings}
\usepackage[ngerman,german]{babel}
\usepackage{german}
\usepackage[utf8]{inputenc}
%\usepackage[latin1]{inputenc}
\usepackage[active]{srcltx}
\usepackage{algorithm}
\usepackage[noend]{algorithmic}
\usepackage{amsmath}
\usepackage{amssymb}
\usepackage{amsthm}
\usepackage{bbm}
\usepackage{enumerate}
\usepackage{graphicx}
\usepackage{ifthen}
\usepackage{listings}
\usepackage{struktex}
\usepackage{hyperref}
\usepackage{color}
\usepackage{subfigure}
\usepackage{xcolor,listings}

\definecolor{javaBlue}{RGB}{42,0.0,255}
\definecolor{javaGreen}{RGB}{63,127,95}
\definecolor{javaPurple}{RGB}{127,0,85}
\definecolor{javaRed}{RGB}{127,0,0}
\lstset{
	keywordstyle=\color{javaPurple},
	commentstyle=\color{javaGreen},
	stringstyle=\color{javaGreen},
	breaklines=true,
	breakautoindent=true, 
	postbreak=\space,  
	tabsize=2,  
	basicstyle=\ttfamily\footnotesize, 
	showspaces=false,       
	showstringspaces=false, 
	extendedchars=true,      
	backgroundcolor=\color{black!10},
	emph={@Autowired,@Controller,@RequestMapping,@ModelAttribute,@Service,@Scope,@Entity,@Table,@Column,@Id}, 
	emphstyle=\color{javaBlue},
	captionpos=b
}

%%%%%%%%%%%%%%%%%%%%%%%%%%%%%%%%%%%%%%%%%%%%%%%%%%%%%%
%%%%%%%%%%%%%% EDIT THIS PART %%%%%%%%%%%%%%%%%%%%%%%%
%%%%%%%%%%%%%%%%%%%%%%%%%%%%%%%%%%%%%%%%%%%%%%%%%%%%%%
\newcommand{\Fach}{Informatik AG}
\newcommand{\Name}{Prof. Dr. Jörg Hettel; Dorian Weidler, B. Sc.}
\newcommand{\Semester}{SS 17}
\newcommand{\Uebungsblatt}{1}
%%%%%%%%%%%%%%%%%%%%%%%%%%%%%%%%%%%%%%%%%%%%%%%%%%%%%%
%%%%%%%%%%%%%%%%%%%%%%%%%%%%%%%%%%%%%%%%%%%%%%%%%%%%%%

\setlength{\parindent}{0em}
\setlength{\parskip}{1em}
\topmargin -1.0cm
\oddsidemargin 0cm
\evensidemargin 0cm
\setlength{\textheight}{9.2in}
\setlength{\textwidth}{6.0in}

%%%%%%%%%%%%%%%
%% Aufgaben-COMMAND
\newcommand{\Aufgabe}[1]{
  {
  %\vspace*{0.5cm}
  \textsf{\textbf{Aufgabe #1}}
  %\vspace*{0.2cm}
  
  }
}

%% Aufgaben-COMMAND
\newcommand{\Head}[1]{
	{
	%	\vspace*{0.5cm}
		\textsf{\textbf{#1}}
	%	\vspace*{0.2cm}
		
	}
}
%%%%%%%%%%%%%%
\hypersetup{
    pdftitle={\Fach{}: Übungsblatt \Uebungsblatt{}},
    pdfauthor={\Name},
    pdfborder={0 0 0}
}

\title{Übungsblatt \Uebungsblatt{}}
\author{\Name{}}

\begin{document}
\thispagestyle{fancy}
\lhead{\sf \large \Fach{} \\ \small \Name{}}
\rhead{\sf \Semester{}}
\vspace*{0.2cm}
\begin{center}
\LARGE \sf \textbf{Übungsblatt \Uebungsblatt{}}
\end{center}
\vspace*{0.2cm}

%%%%%%%%%%%%%%%%%%%%%%%%%%%%%%%%%%%%%%%%%%%%%%%%%%%%%%
%% Insert your solutions here %%%%%%%%%%%%%%%%%%%%%%%%
%%%%%%%%%%%%%%%%%%%%%%%%%%%%%%%%%%%%%%%%%%%%%%%%%%%%%%

\Head{Vorwort}
Um zusätzliche Informationen angezeigt zu bekommen können Sie den dritten Parameter der run-Methode auf \textbf{true} setzen. Der vierte Parameter bezweckt, dass das Layout des Levels zufällig generiert wird.

\begin{lstlisting}[language=Java, frame=single, label={lst:AiRunner}, escapechar=!]
public class MarioAiRunner {
	[...]
	public static void run(	IAgent agent,
													LevelConfig level, 
													!\textcolor{blue}{\textbf{boolean debug}}!, 
													!\textcolor{javaGreen}{\textbf{boolean randomize}}!) {
		[...]
	}
}
\end{lstlisting}

Um, über die hier gestellten Aufgaben hinaus, Ihren Algorithmus weiter zu optimieren, finden Sie weitere Level im Code.

\Aufgabe{1}
Machen Sie sich mit den Funktionen der API vertraut. Dazu können Sie zunächst das Level \textbf{\mbox{LEVEL\_1}} verwenden.

Hier können Sie auf einem flachen Level ohne Gegner und Hindernisse die Grundfunktionen der API erproben und sich mit dem Programm vertraut machen.

\begin{enumerate}[a)]
	\item Lassen Sie Mario dauerhaft nach rechts laufen.
	\item Lassen Sie Mario beim nach rechts laufen springen.
	\item Lassen Sie Mario nach rechts sprinten.
\end{enumerate}

%\clearpage

\Aufgabe{2}
Um es etwas schwerer zu machen treten nun auch noch Schluchten auf, die überwunden werden müssen. Dazu laden Sie \textbf{\mbox{LEVEL\_2}}

\begin{enumerate}[a)]
	\item Ziel ist es möglichst schnell durch das Level zu kommen.
	%\item Ziel ist es möglichst viele Gegner zu erledigen und zugleich möglichst schnell im Ziel zu sein.
	\item Vergleichen Sie Ihr Ergebnis mit denen in der Gruppe.
\end{enumerate}

\Aufgabe{3}
Zum Level aus Aufgabe 2 kommen nun noch Gegner hinzu. Bei den Gegnern handelt es sich um Goombas und Koopas. Laden Sie \textbf{\mbox{LEVEL\_3}}.
\begin{enumerate}[a)]
	\item Ziel ist es möglichst schnell durch das Level zu kommen.
	\item Ziel ist es möglichst viele Gegner zu erledigen und zugleich möglichst schnell im Ziel zu sein.
	\item Vergleichen Sie Ihr Ergebnis mit denen in der Gruppe.
\end{enumerate}

\textbf{Tipp:} Die Gegner in diesem Level können entweder mit Feuerbällen abgeschossen werden oder man kann auf sie springen.

\begin{figure}[H]
	\centering
	\subfigure[Goomba]{\includegraphics[height=0.11\textheight]{bilder/goomba.png}}
	\hspace{30px}
	\subfigure[Koopa]{\includegraphics[height=0.11\textheight]{bilder/koopa.png}}
	\caption{Mögliche Gegner aus \mbox{LEVEL\_3}}
\end{figure}

%\clearpage

\Aufgabe{4}
Laden Sie \textbf{\mbox{LEVEL\_4}} und versuchen Sie dieses Level zu bezwingen. Die Ziele sind gleich denen von Aufgabe 3. In diesem Level kommen aber noch Spikys und Röhren dazu.
\begin{figure}[H]
	\centering
	\subfigure[Goomba]{\includegraphics[height=0.11\textheight]{bilder/goomba.png}}
	\hspace{30px}
	\subfigure[Koopa]{\includegraphics[height=0.11\textheight]{bilder/koopa.png}} 
	\hspace{30px}
	\subfigure[Spiky]{\includegraphics[height=0.11\textheight]{bilder/spiny.png}}
	\caption{Mögliche Gegner aus \mbox{LEVEL\_4}}
\end{figure}

\Aufgabe{5}
Um die Grenzen Ihres Algorithmus auszuloten wird nun die Umgebung erheblich komplexer. Es kommen verschiedene Blöcke hinzu und außerdem auch PowerUps. Zum Einstieg sind noch keine Gegner aktiviert. Laden Sie \textbf{\mbox{LEVEL\_5}}

\Aufgabe{6*}
Zum Abschluss dieser Übung erproben Sie Ihren mittlerweile fortgeschrittenen Algorithmus noch am schwierigsten der Level. Es ist analog zum Level aus Aufgabe 5*, hier treten aber zusätzlich alle Gegner aus den vorigen Aufgaben auf. Laden Sie dazu Level \textbf{\mbox{LEVEL\_6}}.

\begin{figure}[H]
	\centering
	\subfigure[Goomba]{\includegraphics[height=0.11\textheight]{bilder/goomba.png}}
	\hspace{30px}
	\subfigure[Koopa]{\includegraphics[height=0.11\textheight]{bilder/koopa.png}} 
	\hspace{30px}
	\subfigure[Spiky]{\includegraphics[height=0.11\textheight]{bilder/spiny.png}}
	\caption{Mögliche Gegner aus \mbox{LEVEL\_6}}
\end{figure}

%%%%%%%%%%%%%%%%%%%%%%%%%%%%%%%%%%%%%%%%%%%%%%%%%%%%%%
%%%%%%%%%%%%%%%%%%%%%%%%%%%%%%%%%%%%%%%%%%%%%%%%%%%%%%
\end{document}
